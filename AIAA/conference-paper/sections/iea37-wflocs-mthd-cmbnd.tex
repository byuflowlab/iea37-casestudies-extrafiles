The intent of this Case Study is to assess not only the optimization methods measured by Case Study 1, but also the effects that different physics model approximations have on turbine location recommendations.

Case Study 2 differs from the previous one in that 1) no wake model is provided, and 2) only a single wind farm size is to be optimized.
Participants are free to chose their preferred EWM and optimization method combination.

Unlike Case Study 1, participant reported AEP is not comparable, since different EWMs (which account for different physics phenomena) are used to calculate them.
To help with this, we conducted a cross-comparison of results between participants.
For the cross-comparison, each participant's proposed optimal turbine locations in the standardized \texttt{.yaml} format was published to the other Combined Case Study participants.
Each participant then used their own wake model to calculate the AEP of the other participant's proposed farm layouts.
From this portion of the Case Study, we hope to learn if any participant's results are seen as superior by other EWMs.

\subsubsection{Farm Attributes}
The wind farm size for the Combined Case Study is limited to 9 turbines.
We did this to limit the computation time requirements when assessing results in a standardized LES, discussed later in \cref{sec:ftr-wrk}.
We used the previously described method under \textbf{Farm Diameter} to determine the boundary radius, and the wind rose and wind speed are the same as Case Study 1.