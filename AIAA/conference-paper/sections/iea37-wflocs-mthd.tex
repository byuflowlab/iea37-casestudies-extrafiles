			
	% This is how we did it

Two of the major factors contributing to superior turbine placement recommendations are
1) EWM characteristics and 2) optimization algorithm.
We have therefore designed two distinct case studies in an attempt to quantify the effects of alterations in each of these variables.

To isolate variability in the second factor (optimization method), we pre-coded a representative wake model as a control variable,
and permit participants to use any optimization strategy they think will achieve the best Annual Energy Production (AEP) for the farm.
This first scenario is called the Optimization Only Case Study, and is described below in \cref{sec:optonly}.

Isolating the other factor (EWM variability) proves more complicated.
An EWM's compatibility with gradient-based or gradient-free optimization methods dictate which algorithms can be applied.
As such, designing a case study that restricts participants to a single optimization algorithm would unnecessarily limit the scope of EWMs studied.
With the aim of acquiring as much empirical data as possible in order to determine best practices for the industry as a whole,
our second case study permits participant selection of not only EWM, but also implemented optimization algorithm.
It is called the Combined Physics Model/Optimization Algorithm Case Study, and is described below in \cref{sec:cmbnd}.

To enable production of useful data, both case studies require a model wind farm with characteristics which are simultaneously restrictive enough to maintain simplicity, yet general enough to maintain relevance for more complex and realistic problems.
The wind farm scenarios selected for the case studies are described below in \cref{sec:windfarm}
\bigskip
\subsection{Common to Both Case Studies} \label{sec:windfarm}

	\import{./sections/}{iea37-wflocs-mthd-bothcs.tex}
	
\subsection{Optimization Only Case Study} \label{sec:optonly}

	\import{./sections/}{iea37-wflocs-mthd-optonly.tex}

\subsection{Combined Physics Model/Optimization Algorithm Case Study} \label{sec:cmbnd}

	\import{./sections/}{iea37-wflocs-mthd-cmbnd.tex}