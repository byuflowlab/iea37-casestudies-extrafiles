% This is what we did

\subsection{Case Study 1: Optimization Only}

Each participant ran the optimization algorithm and implementation of their choosing using our suppied AEP target function, or an equivalent duplicaiton in another language.
Since there exists a great deal of variability in hardware, participants also reported processor speed, function calls, number of cores utilized, and amount of RAM installed their system when finding their optimized results.
The AEP results and rankings are given below in \cref{tab:results}

There were 10 submission for the Optimization Only Case Study.
One participant submitted twice, using a different optimization method for each submission.
These two submissions will be hereafter treated as if they were from different participants, and are assigned different participant numbers.
For anonimity, each submission is given a number.
We will referred to each submission below by this number (i.e. participant 1, ..., participant 10, etc.).

\subsubsection{16 Turbine Case}

	Though wall time and number of function calls varied widely, the best 4 (of 10) participant submissions used gradient-based algorithms.

	\begin{table}[H]
		\begin{center}
			\begin{tabular}{r l l c}
				\hline
				Rank           & AEP         & Participant \# & Gradients \\
				\hline
				1              & 418924.4064 & 4              & Yes       \\
				2              & 412251.1945 & 8              & Yes       \\
				3              & 414141.2938 & 5              & Yes       \\
				4              & 411182.2200 & 1              & Yes       \\
				5              & 409689.4417 & 2              & No        \\
				6              & 408360.7813 & 10             & Yes       \\
				7              & 402318.7567 & 3              & No        \\
				8              & 392587.8580 & 7              & No        \\
				9              & 388758.3573 & 6              & No        \\
				10             & 388342.7004 & 9              & No        \\
				example layout & 366941.5712 & 11             & N/A       \\
				\hline
			\end{tabular}
		\end{center}
		\caption{Participant results of the 16 turbine scenario}
		\label{tab:results}
	\end{table}

\subsubsection{36 Turbine Case}

	\begin{table}[H]
		\begin{center}
			\begin{tabular}{r l l c}
				\hline
				Rank           	& AEP         	& Participant \# & Gradients \\
				\hline
				1				& 863676.2993	& 4				& Yes       \\
				2				& 851631.931	& 10			& Yes       \\
				3				& 849369.7863	& 2             & No        \\
				4				& 846357.8142	& 8             & Yes       \\
				5				& 844281.1609	& 1             & Yes       \\
				6				& 828745.5992	& 3             & No        \\
				7				& 813544.2105	& 9             & No        \\
				8				& 777475.7827	& 7             & No        \\
				9				& 776000.1425	& 6             & No        \\
				example layout	& 737883.0985	& 11            & N/A        \\
				\hline
			\end{tabular}
		\end{center}
		\caption{Prticipant results of the 36 turbine scenario}
		\label{tab:results}
	\end{table}

\subsubsection{64 Turbine Case}

	\begin{table}[H]
		\begin{center}
			\begin{tabular}{r l l c}
				\hline
				Rank           	& AEP         	& Participant \# & Gradients \\
				\hline
				1& 	1513311.194	& 4 & Yes       \\
				2& 	1506388.415	& 2& No        \\
				3&	1480850.976	& 10& Yes       \\
				4&	1476689.663& 1& Yes       \\
				5& 	1455075.608	& 3& No        \\
				6&	1425678.143	& 8& Yes       \\
				7&	1422268.714	& 9& No        \\
				8& 	1364943.008	& 6& No        \\
				9&	1332883.433	& 7& No        \\
				example layout	& 1294974.298&	11 \\
				\hline
			\end{tabular}
		\end{center}
		\caption{Prticipant results of the 64 turbine scenario}
		\label{tab:results}
	\end{table}

\subsection{Case Study 2: Combined}

\todo[inline, color=red!40]{Not enough time/resources for LES right now.}

\todo[inline, color=red!40]{Cross-Comparison of results was conducted.}

%---- Eduardo's stuff --%
%As this research progresses, a validation of the propeller-on-propeller interactions predicted by VPM will be performed in three phases:

%\begin{enumerate}
%	\item Modeling of the exact experimental setup used in the PIV measurements reported by Zhou \textit{et al.}\cite{Zhou2017} and comparison of the predicted velocity field of this two co-rotating propellers.
%	\item Sweeping of separation distance between the two co-rotating propellers used by Zhou \textit{et al.} and comparison between measured and predicted aerodynamic performance (thrust and torque).
%	\item Once validity is established, we will perform a parametric study of performance on APC 10x7 propellers interacting at varying advance ratios, Reynolds numbers, and separation distance, in counter and co-rotation configurations.
%\end{enumerate}

%With the development and validation of the method presented in this study we aim to show the capabilities of the VPM to model propeller-on-propeller interactions in a first-principles-based approach, with an accuracy and speed well fit for the conceptual design of distributed-propulsion aircraft.