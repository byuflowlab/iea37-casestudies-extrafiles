% This is what we did
% Overview
%	numbers
%	gradient/non gradient participants
% Winners
%	Best Gradients


\subsection{Case Study 3: Concave Boundary}\label{sec:res-optonly}


%	\import{./sections/}{iea37-wflocs-rslts-optonly.tex}

\subsection{Case Study 4: Disjoint Boundary}\label{sec:res-cmbnd}


%	\import{./sections/}{iea37-cs34-rslts-cmbnd.tex}

%---- Eduardo's stuff --%
%As this research progresses, a validation of the propeller-on-propeller interactions predicted by VPM will be performed in three phases:

%\begin{enumerate}
%	\item Modeling of the exact experimental setup used in the PIV measurements reported by Zhou \textit{et al.}\cite{Zhou2017} and comparison of the predicted velocity field of this two co-rotating propellers.
%	\item Sweeping of separation distance between the two co-rotating propellers used by Zhou \textit{et al.} and comparison between measured and predicted aerodynamic performance (thrust and torque).
%	\item Once validity is established, we will perform a parametric study of performance on APC 10x7 propellers interacting at varying advance ratios, Reynolds numbers, and separation distance, in counter and co-rotation configurations.
%\end{enumerate}

%With the development and validation of the method presented in this study we aim to show the capabilities of the VPM to model propeller-on-propeller interactions in a first-principles-based approach, with an accuracy and speed well fit for the conceptual design of distributed-propulsion aircraft.