%-- This is so far unaltered from the extended abstract you submitted --%

\begin{abstract}
	\textbf{This paper presents the results of four case studies regarding the wind farm layout optimization problem.
		We asked members of the computational optimization and wind communities to take part in the studies that we designed. \textbf{Nine} individuals participated.
		Case study 1 considered variations in optimization strategies for a given simple Gaussian wake model.
		Participants were provided with a wake model that outputs annual energy production (AEP) for an input set of wind turbine locations.
		Participants used an optimization method of their choosing to find an optimal wind farm layout.
		Case study 2 looked at trade-offs in performance resulting from variation in both physics model and optimization strategy.
		For case study 2, participants calculated AEP using a wake model of their choice while also using their chosen optimization method.
		%Participant submissions for optimized turbine locations were then compared to Large Eddy Simulator (LES) calculations for AEP, and results were measured for both quality and accuracy.
		%Participant submissions for optimized turbine locations were then cross-compared by recalculating the AEP using every other participant's wake model.
		Participants then used their wake model to calculate the AEP of all other participants' optimized layouts.
		Results for case study 1 show that the best optimal wind farm layouts in this study were achieved by participants who used gradient-based optimization methods.
		A front-runner emerged with the Sparse Nonlinear OPTimizer plus Wake Expansion Continuation (SNOPT\texttt{+}WEC) optimization method, which consistently discovered the highest submitted AEP.
		%The results for case study 2 show that, for small wind farms with few turbines, turbine placement on the wind farm boundary is superior.
		For case study 2, two participants found a similar layout that was judged to be superior by all five participants.  It is unclear if the better solution resulted from an improved optimization process, or a wake model that was more amenable to optimization.
		}
\end{abstract}